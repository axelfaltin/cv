\begin{rubric}{Profile}

\entry*[A Key]
  This is an entry with a key. The key is displayed on the left,
  and you're reading the entry's contents. As you can see, this entry does
  not belong to a subrubric.

    \subrubric{A First Subrubric}

\entry*[A Key]
  This entry belongs to the first subrubric. Before the subrubric,
  some space is added to separate it from the previous entry.
\entry*
  After the subrubric, some space is also added to separate it from the
  first entry. Note that this entry has no key. The entries contents are
  aligned together.
\entry*[Another Key]
  This is another entry with a new key.
\entry*
  This is another entry, but this one has no key. Note the text bullet
  which serves as a visual clue, especially when several entries share the
  same key.

    \subrubric{A Second Subrubric}

\entry*[A Key]
  This entry belongs to the second subrubric.
\entry*
  This one also belongs to the second subrubric.
\entry*[Another Key]
  This is another entry with a new key.
\entry*
  This is another entry, but this one has no key.

    \subrubric{}

\entry*[A Key]
  If you want to separate some entries from the subrubric above,
  you can for instance make an empty subrubric.
\entry*
  You can include other rubrics below. Rubrics can even be split across
  pages. The titles will then be repeated.

\end{rubric}
