\begin{rubric}{Berufserfahrung}
\subrubric{TAKKT Group - Stuttgart}
\entry*[01/2022 - heute] \textbf{Executive VP Technology, Data \& Security: CIO}\newline
\textax{Inhalte:} 
Leitung der Gruppenfunktion Technology (EU \& USA), Transformationsstrategie der TAKKT Gruppe von Einzelunternehmen in einen integrierten Konzern: Integration von 15 Business Units in einen gemeinsamen Shared Service: Umbau technische Bebauung, Personal, Infrastruktur und Geschäftsprozesse, stetige Optimierung.

\textax{Rolle:} 
CIO, CISO, CTO, Berichtet an CEO TAKKT AG, Gesamtverantwortung Technology, Data und Security, 180 MA, 100 Mio Budgetverantwortung, 1,2 Mrd Umsatz

\entry*[04/2018 - 01/2022] \textbf{Geschäftsführender CIO}\newline
\textax{Inhalte:} 
Operative Geschäftsführung als Teil des Executives Managenent Teams des Omnichannel Commerce Segments der TAKKT AG

\textax{Rolle:} 
CIO, CISO, CTO, Berichtet an Vorstand TAKKT AG, Gesamtverantwortung der Europ"aischen Technology Kompetenz für die Marken: KAISER+KRAFT, Gerdmann, Ratioform, uw. 100 MA, 60 Mio Budget, 700 Mio Umsatz\newline

\subrubric{Tesla - Palo Alto USA}
\entry*[12/2016 - 04/2018] \textbf{Direktor: Office of the CIO}\newline
\textax{Inhalte:} 
Programmmanagment f"ur alle IT Programme und Projekte, Personalf"uhrung aller Projekt Manager. F"uhrung des IT-Einkaufs und Abwicklung. 

\textax{Rolle:} 
Chief of Staff, Berichtet an CIO, Kaufm"annische Verantwortung IT Budget, Spezialprojekte 
\pagebreak
\subrubric{SolarCity - San Mateo USA}
\entry*[08/2015 - 12/2016] \textbf{Senior Direktor: Technology / Security}\newline
\textax{Inhalte:} 
F"uhrung der IT-Infrastruktur (Rechenzentrum, Compute \& Storage, WAN, LAN) und Devops Teams f"ur alle In-House Anwendungen von Tesla Energy. Gesamtverantwortung f"ur IT-Sicherheit und IT-Compliance f"ur Tesla Energy.

\textax{Rolle:} 
Stabilisierung IT-Infrastruktur und Betrieb, Automatisierung, CI/CD Pipeline, Change \& Release management. Risikoanalyse, Penetration Testing, Vulnerbility Management.\newline

\subrubric{Business Transformation: Crate\&Barrel - Chicago USA}
\entry*[04/2013 - 08/2015] \textbf{Direktor: Technology}\newline
\textax{Inhalte:} 
Gesamtverantwortung IT-Infrastruktur (Disziplinarische F"uhrung von 35 Mitarbeitern/ Budgetverantwortung ca. 10 Mil. USD/Jahr): Rechenzentren (24/7 Betrieb), Lager- und Distributionsstandorte, ca. 100 Station"argesch"afte; Arbeitsplatzsysteme, Support, Server (IBM Power 7+ \& X86, Win/AIX/ISeries), Netzwerk (WAN, LAN, Wi-Fi) und Telko. Projektmanagement f"ur alle Transformationsprojekte im Infrastrukturbereich.

\textax{Rolle:} 
Turnaroundmanagement IT-Infrastruktur, Berater CTO \& CIO: Formulieren, Vorantreiben und Weiterentwicklen der Infrastrukturstrategie. Konsolidierieng, Virtualisierung und Outsourcing: Selektion von Technologie, Intergrationspartnern und Betriebsmodell. Entwickeln und umsetzen einer Cloud-Computing Strategie.

\entry*[10/2011 - 03/2013] \textbf{Direktor: Portfolio Management und Governance}\newline
\textax{Inhalte:} 
Verantwortung und Durchf"uhrung aller Investment- und Programmmanagment-Prozesse (regelm"a"sige Entscheidungsmeetings auf Top-Management-Ebene, Projektanbahnungswesen). Portfolio Management s"amtlicher Projektvorhaben im Konzern (ca. 40 Mil. USD/Jahr). Projektmanagement f"ur einzelne Transformationsprojekte.

\textax{Rolle:} 
Turnaroundmanagement Business und IT, Berater CTO \& CEO: Formulieren, Vorantreiben und Weiterentwicklen der Transformationsstrategie (u. a. von Inhouseentwicklung zur Paket- \& Cloudstrategie, Outsourcing), Vendor-/Partnerauswahl, Programmmanagment, Projektmanagement, Changemanagement.\newline

\subrubric{IT/Logistik: Ottogroup [01/2005-10/2011]}
\entry*[08/2009 - 10/2011] \textbf{Abteilungsleitung: Corporate IT / Governance}\newline
\textax{Inhalte:} 
Verantwortung der Ottogroup Governance-Instrumente: Projektinitialisierung, Projektdefinition, Projektabschluss und Projektnachbetrachtung, Etablierung und Durchf"uhrung des Governance-Executive-Board in einer dezentalen Organisation.\axelvspace

\textax{Rolle:} 
Quality Gate, explizite Freigabe aller IT-Projekte der Ottogroup im Multi-Channel-Einzelhandel. "Uberpr"ufung der Projekte auf Basis einer Projektvereinbarung(PID/PDD) auf Vollst"andigkeit, Plausibilit"at, Wirtschaftlichkeit und IT-Strategiekonformit"at. Beratestab des CIO. Formulierung und Weiterentwicklung der IT-Strategie und des IT-Bebauungsplans. Vorbereitung, Durchf"uhrung und Nachbereitung eines monatlichen Governance-Executive-Boards (Top-Management der Ottogroup). Etablierung der Trennung von IT-Dienstleistung und Corporate IT.

\entry*[08/2006 - 07/2009] \textbf{Abteilungsleitung: Abwicklungs- und Fakturiersysteme}\newline
\textax{Inhalte:} 
Verantwortung fachlicher Prozesse Auftragsanlage bis zur Abgabe an die Lagerstandorte: Bestandszuteilung,
Lieferungsbildung, Lieferungsoptimierung, Abwicklungswegermittlung, Beilagensteuerung, Kapazit"atssteuerung, Rechnungsbildung, Fakturierung, Lagernachschub und Lagerabwicklung.\axelvspace

\textax{Rolle:} 
Abteilungsleitung mit 40 Mitarbeitern (Disziplinarische F"uhrung von 18 Mitarbeitern) an verteilten Standorten. Steuerung externer Dienstleister, Verantwortung f"ur Projektvolumen von ca. 7 Mio EUR pro Jahr.

\entry*[01/2005 - 07/2006] \textbf{Entwicklungsleitung: Bestands- und Artikelsysteme}\newline
\textax{Inhalte:} 
Online Anbindung eines Hostsystems an eine J2EE Architektur im Rahmen logistischer Prozesse: Sortierzielermittlung f"ur die Retourensteuerung\axelvspace

\textax{Rolle:} 
Fachliche F"uhrung der Entwicklungsmannschaft (15 Mitarbeiter), Steuerung externer Dienstleister, Releasemanagement und Lieferverantwortung f"ur ein ca. 4000 Personentage Softwarerelease.\newline

\subrubric{Consulting: Accenture [12/2001-12/2004]}
\entry*[10/2003 - 12/2004] \textbf{Investment Banking: Securities Processing Solution}\newline
\textax{Projekt:} 
Funktionales/technisches Design, Umsetzung und Test eines Handelsgesch"afts-Anreicherungssystems f"ur den institutionellen Wertpapierhandel einer deutschen Gro"sbank.\axelvspace

\textax{Rolle:} 
Als technischer Architekt des Projekts und Teamlead von neun Softwareentwicklern war ich f"ur das technische Design und die Programmierung der Kernfunktionalit"at Gesch"aftsanreicherung und -verarbeitung verantwortlich. 
%\axelvspace
%\textax{Technische Architektur:}
%Java, ant, CVS, IBM Enterprise Application Integration Middleware, Bea Weblogic J2EE Application Server, Oracle Toplink Persistenz, Oracle 9i Datenbank

\entry*[07/2003 - 10/2003] \textbf{Capital Markets: Transatlantische Handels- \& Clearing Plattform}\newline
\textax{Projekt:} 
Entwurf eines funktionen Gesamtmodells f"ur das Clearing von Finanzkontrakten, gehandelt an B"orsen in den USA und Europa. Eine der gro"sen Herausforderungen waren die Unterschiede in der Clearing Infrastruktur und die Anforderungen der Zulassungsbeh"orden der verschiedenen Staaten.\axelvspace

\textax{Rolle:} 
Als Mitglied des Business Architektenteam wurden Analysen der Infrastruktur, der legalen sowie der organisatorischen Anforderungen durchgef"uhrt. Dar"uberhinaus lag meine Verantwortung in der Ein- und Durchf"uhrung von Projekt Management Prozessen wie z. B. eines Issue Management Prozesses.

%\entry*[06/2003 - 07/2003] \textbf{Personalf"orderung und Training: Core Analyst School}\newline
%\textax{Projekt:} 
%Durchf"uhrung des Orientierungstraining (in englischer Sprache) f"ur neue Mitarbeiter im zentralen Trainingszentrum von  Accenture in St. Charles, Illinois, USA.\axelvspace

%\textax{Rolle:} 
%Meine Aufgabe bestand darin ein zweiw"ochiges Orientierungstraining vorzubereiten und als Lehrer, Manager und Coach durchzuf"uhren. Als Teil eines vierk"opfigen Teams war ich verantwortlich fuer das Training von 40 Teilnehmer aus allen Teilen der Erde inklusive formaler Beurteilungen der Teilnehmer.
% Echte Zeitachse \entry*[05/2002 - 06/2003] \textbf{Capital Markets: Central Counterparty f"ur Equities (CCP)}\newline

\entry*[12/2001 - 06/2003] \textbf{Capital Markets: Central Counterparty f"ur Equities (CCP)}\newline
\textax{Projekt:} 
Eine deutsche B"orse, die bereits als zentraler Kontrahenten f"ur Derivate, Renten und Repos am Markt t"atig war, trat in den Aktien-Kassa-Markt ein. %F"ur die Gesch"afte in die der Kontrahent automatisch eintritt werden zur Abdeckung des Marktrisikos netto Risikopositionen f"ur jeden Marktteilnehmer ermittelt und die Besicherungsanforderungen berechnet und dargestellt.\axelvspace

\textax{Rolle:} 
In meiner Verantwortung lag die Erstellung von technischen Designs auf Grundlage funktionaler Anforderungen f"ur die Benutzeroberfl"ache der Markt- \& Clearingaufsicht. Nach Produktionseinf"uhrung leitete ich das gesamte GUI Teams aus Kunden und Accenture Mitarbeitern. Hauptaufgaben umfassten Planung, zeitliche Einteilung und "Uberwachung von Fixits und Change Request. \newline
%\axelvspace
%\textax{Technische Architektur:} 
%Java, ant, CCC Harvest, Bea Weblogic J2EE Application Server, VMS Host, Oracle RDB
%\entry*[09/2002] \textbf{Bef"orderung vom Analyst zum Consultant}\newline
%\entry*[03/2002 - 05/2002] \textbf{Banking: Apache Caching Module}\newline
%\textax{Projekt:} 
%Entwicklung eines Apache dynamic module (DSO) zur Zwischenspeicherung von pseudo-dynamischen Inhalten in einem Bankenportal.
%\axelvspace
%\textax{Rolle:} 
%Nach dem Aufsetzen der Entwicklungsumgebung war ich verantwortlich f"ur Design, Implementierung, Test und Dokumentation des ersten Release auf Basis eines konzeptionellen Prototypen.
%\axelvspace
%\textax{Technische Architektur:} 
%Apache, UNIX, C, CVS, make
%\entry*[12/2001 - 02/2002] \textbf{Banking: Projekt Review}\newline
%\textax{Projekt:} 
%Es galt ein gro"ses mehrj"ahriges IT Infrastruktur Projekt der gr"o"sten Retailbank Deutschlands zu untersuchen, um insbesondere qualit"atssicherungs- und risikoreduzierende Ma"snahmen %vorzuschlagen und dem Vorstand zu pr"asentieren.
%\axelvspace
%\textax{Rolle:} 
%Als Teil des Review Teams, welches die Vorbereitung, die zeitliche Planung und die Durchf"uhrung von strukturierten Interviews inne hatte, waren meine Hauptaufgaben: Ermitteln und isolieren von %Risiken und Issues im Projekt auf Basis der durchgef"uhren Interviews sowie die Vorbereitung der Abschlu"spr"asentation f"ur den Vorstand.  

%\subrubric{Logistik: Otto Versand}
%\entry*[01/2001 - 10/2001] \textbf{Software Entwicklung: Lagerverwaltung}\newline
%\textax{Projekt:} 
%Entwicklung einer Lagerverwaltungs-Software in einer verteilten, mehrere Betriebs\-systeme (Windows, Linux, Solaris) umfassenden Umgebung.\axelvspace

%\textax{Rolle:} 
%Meine Aufgabe bestand darin, Tagesend-Verarbeitungen f"ur die eingesetzte propriet"are Enterprise Resource Planning (ERP) Software zu entwerfen und zu programmieren. Dar"uberhaus lag meine Verantwortung in der Qualit"atssicherung, dem Test und dem Rollout des Systems inklusive Anwendertraining.\newline
%\axelvspace
%\textax{Technische Architektur:} 
%UNIX, Windows, C++, Java, Oracle, tsh, bash, \LaTeX

%\subrubric{"Offentlicher Dienst: Stadt Wedel}
%\entry*[09/1997 - 05/2001] \textbf{Unterricht: Volkshochschulkurse}\newline
%\textax{Projekt:} 
%Einf"uhrungskurse in die Informationstechnologie (Basiskurse PC, Word, Excel, Java) als Abend- und Wochenendkurse an der Volkshochschule Wedel.\axelvspace

%\textax{Rolle:} 
%Die Aufgabe bestand darin die Kurse vorzubereiten und geeignet durchzuf"uhren. Die Gruppenst"arke lag bei 16 Personen aus allen Gesellschaftsschichten, die teilweise einer intensiver Betreuung und Motivation bedurften. 

%\subrubric{Software Entwicklung: iXL Consulting}
%\entry*[06/1999 - 10/1999] \textbf{Reiseverantstaltung: Websitelaunch}\newline
%\textax{Projekt:} 
%Einf"uhrung einer transaktionalen Website f"ur einen Reiseverantstalter inklusive Online Buchung, Zahlung und Newsletter Services.\axelvspace

%\textax{Rolle:} 
%Meine Aufgaben umfassten die technische Vorbereitung der Einf"uhrung: Organisation und Installation der Hard- \& Software sowie die Entwicklung von Komponenten zur Abwicklung von Buchungen\newline %und Email-Benachrichtungen.
%\axelvspace
%\textax{Technische Architektur:} 
%Apache, Solaris, Java, Informix
%\subrubric{Software Development: Agens Consulting}
%\entry*[06/1999 - 06/1999] \textbf{Value benefit analysis tool}\newline
%\textax{Projekt:} 
%Redesign and portation of a tool to undertake value benefit analysis from %Smalltalk to Java. Responsibilities were focused on the design phase.  

%\axelvspace
%\textax{Rolle:} 

%\subrubric{Bankwesen: Vereins- und Westbank }
%\entry*[06/1997 - 05/1999] \textbf{Controlling:}\newline
%\textax{Projekt:} 
%Controlling, Reporting und Planung f"ur 180 Bankfilialen (09/1998 - 05/1999)\newline
%1st und 2nd Level Benutzerservice (06/1997 - 09/1998)\axelvspace

%\textax{Rolle:} 
%Als Teil eines Teams war ich mitverantwortlich f"ur die monatliche Profit-Center-Ergebnisrechnung und dessen Benchmarking. Dar"uberhinaus bestand meine Aufgabe in der Umsetzung von neuen Reports und die Mitentwicklung an einem papierlosen Reportingsystem.
%Als Mitglied des Benutzerservice Teams war ich im Second-Level Support des User-Help-Desk t"atig. Dar"uberhinaus bestand meine Aufgabe in der Entwicklung von Software-Komponenten zur Vereinfachung und Automatisisierung von Routineaufgaben.
\sloppy
\end{rubric}
